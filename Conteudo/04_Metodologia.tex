\chapter{Metodologia}

Breve descrição das próximas subseções...

\subsection{Questões de pesquisa}

Descrição de todas as perguntas que se pretende responder ao final deste trabalho... 

A seguir as questões de pesquisa:
\begin{enumerate}
	\item Questão de pesquisa 01;
	\item Questão de pesquisa 02;
	\item Questão de pesquisa 03;
	\item Questão de pesquisa 04;
	\item Questão de pesquisa n.
\end{enumerate}

\subsection{Estratégia de busca}

Descrição das estratégias de busca, informando as bases e as strings utilizadas...

Foram utilizadas as seguintes bases de pesquisa:
\begin{itemize}
	\item BASE \url{<https://base-search.net/about/en/index.php>};
	\item ERIC	\url{<https://eric.ed.gov/>};
	\item IEEE Xplore Digital Library	\url{<http://ieeexplore.ieee.org>};
	\item SciELO	\url{<https://search.scielo.org/>};
	\item Science Direct	\url{<https://www.sciencedirect.com/>}.
\end{itemize}


Na Tabela \ref{palavra_chave} são apresentadas as Palavras-Chave utilizadas utilizadas para formar a \textit{string} de busca.
Para auxiliar na criação de tabelas pode-se utilizar o site \url{https://www.tablesgenerator.com/}

\begin{table}[ht]
	\centering
	\label{palavra_chave}
	\caption{Palavras-Chave utilizadas na \textit{string} de busca}
	\begin{tabular}{|c|c|}
		\hline
		\textbf{\begin{tabular}[c]{@{}c@{}}Palavra chave\end{tabular}}                             & \textbf{\begin{tabular}[c]{@{}c@{}}Sinônimo em Inglês\end{tabular}} \\ \hline
		café                                                                                           & Coffee, cafe                                                            \\ \hline
		\begin{tabular}[c]{@{}c@{}}máquina (Referente a cafeteira ou máquina de café)\end{tabular} & machine, make, maker                                                    \\ \hline
		portátil                                                                                       & portable, mini, hand, tiny                                              \\ \hline
	\end{tabular}
\end{table}



Na Tabela \ref{string} é apresentada a \textit{string} utilizada para as buscas nas bases:

\begin{table}[h!]
	\centering
	\label{string}
	\caption{ \textit{String} utilizada para realizar as buscas nas bases}
	\begin{tabular}{|l|}
		\hline
		\textbf{\begin{tabular}[c]{@{}l@{}}( ( coffee  OR  cafe )  AND  ( machine  OR  make  OR  maker )\\   AND   ( portable  OR  mini  OR  hand  OR  tiny ) )\end{tabular}} \\ \hline
	\end{tabular}
\end{table}


A seguir os Critérios de Inclusão:
\begin{enumerate}
	\item Critério 01;
	\item Critério 02;
	\item Critério 03;
	\item Critério 04;
	\item Critério n.
\end{enumerate}


A seguir os Critérios de Exclusão:
\begin{enumerate}
	\item Critério 01;
	\item Critério 02;
	\item Critério 03;
	\item Critério 04;
	\item Critério n.
\end{enumerate}
